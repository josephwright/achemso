% \iffalse meta-comment
%
% Portions copyright (C) 1989-2003 by
%    Donald Arseneau
% Copyright (C) 2007-08 by
%    Joseph Wright <joseph.wright@morningstar2.co.uk>
%
% Part of this file is derived from cite.sty, to which the following
% license applies:
%--------------------------------------------------------------------
%     Copyright (C) 1989-2003 by Donald Arseneau
%     These macros may be freely transmitted, reproduced, or modified
%     provided that this notice is left intact.
%--------------------------------------------------------------------
%
% This work may be distributed and/or modified under the
% conditions of the LaTeX Project Public License, either
% version 1.3c of this license or (at your option) any later
% version. The latest version of this license is in
%    http://www.latex-project.org/lppl.txt
% and version 1.3c or later is part of all distributions of
% LaTeX version 2005/12/01 or later.
%
% This work has the LPPL maintenance status `maintained'.
%
% The current maintainer of this work is Joseph Wright.
%
% This work consists of the source file natmove.dtx
%                 and the derived files natmove.ins,
%                                       natmove.sty,
%                                       natmove.pdf,
%                                       natmove-manual.pdf and
%                                       jawltxdoc.sty
%
% TDS-ready files:
%    The compressed file achemso.tds.zip contains an unpacked version
%    of all of the files included here, and pre-compiled
%    documentation in PDF format.  Simply decompress achemso.tds.zip
%    in your local TeX directory, run your hash program (texhash,
%    initexmf --update-fndb, etc.) and everything will be ready to
%    go.  The user documentation for the package is called
%    natmove-manual.pdf; the file natmove.pdf includes the user
%    manual and the fully-indexed source code.
%
% Unpacking:
%    (a) If natmove.ins is present:
%           tex natmove.ins
%    (b) Without natmove.ins:
%           tex natmove.dtx
%    (c) If you use LaTeX to generate files:
%           latex \let\install=y% \iffalse meta-comment
%
% Portions copyright (C) 1989-2003 by
%    Donald Arseneau
% Copyright (C) 2007-08 by
%    Joseph Wright <joseph.wright@morningstar2.co.uk>
%
% Part of this file is derived from cite.sty, to which the following
% license applies:
%--------------------------------------------------------------------
%     Copyright (C) 1989-2003 by Donald Arseneau
%     These macros may be freely transmitted, reproduced, or modified
%     provided that this notice is left intact.
%--------------------------------------------------------------------
%
% This work may be distributed and/or modified under the
% conditions of the LaTeX Project Public License, either
% version 1.3c of this license or (at your option) any later
% version. The latest version of this license is in
%    http://www.latex-project.org/lppl.txt
% and version 1.3c or later is part of all distributions of
% LaTeX version 2005/12/01 or later.
%
% This work has the LPPL maintenance status `maintained'.
%
% The current maintainer of this work is Joseph Wright.
%
% This work consists of the source file natmove.dtx
%                 and the derived files natmove.ins,
%                                       natmove.sty,
%                                       natmove.pdf,
%                                       natmove-manual.pdf and
%                                       jawltxdoc.sty
%
% TDS-ready files:
%    The compressed file achemso.tds.zip contains an unpacked version
%    of all of the files included here, and pre-compiled
%    documentation in PDF format.  Simply decompress achemso.tds.zip
%    in your local TeX directory, run your hash program (texhash,
%    initexmf --update-fndb, etc.) and everything will be ready to
%    go.  The user documentation for the package is called
%    natmove-manual.pdf; the file natmove.pdf includes the user
%    manual and the fully-indexed source code.
%
% Unpacking:
%    (a) If natmove.ins is present:
%           tex natmove.ins
%    (b) Without natmove.ins:
%           tex natmove.dtx
%    (c) If you use LaTeX to generate files:
%           latex \let\install=y% \iffalse meta-comment
%
% Portions copyright (C) 1989-2003 by
%    Donald Arseneau
% Copyright (C) 2007-08 by
%    Joseph Wright <joseph.wright@morningstar2.co.uk>
%
% Part of this file is derived from cite.sty, to which the following
% license applies:
%--------------------------------------------------------------------
%     Copyright (C) 1989-2003 by Donald Arseneau
%     These macros may be freely transmitted, reproduced, or modified
%     provided that this notice is left intact.
%--------------------------------------------------------------------
%
% This work may be distributed and/or modified under the
% conditions of the LaTeX Project Public License, either
% version 1.3c of this license or (at your option) any later
% version. The latest version of this license is in
%    http://www.latex-project.org/lppl.txt
% and version 1.3c or later is part of all distributions of
% LaTeX version 2005/12/01 or later.
%
% This work has the LPPL maintenance status `maintained'.
%
% The current maintainer of this work is Joseph Wright.
%
% This work consists of the source file natmove.dtx
%                 and the derived files natmove.ins,
%                                       natmove.sty,
%                                       natmove.pdf,
%                                       natmove-manual.pdf and
%                                       jawltxdoc.sty
%
% TDS-ready files:
%    The compressed file achemso.tds.zip contains an unpacked version
%    of all of the files included here, and pre-compiled
%    documentation in PDF format.  Simply decompress achemso.tds.zip
%    in your local TeX directory, run your hash program (texhash,
%    initexmf --update-fndb, etc.) and everything will be ready to
%    go.  The user documentation for the package is called
%    natmove-manual.pdf; the file natmove.pdf includes the user
%    manual and the fully-indexed source code.
%
% Unpacking:
%    (a) If natmove.ins is present:
%           tex natmove.ins
%    (b) Without natmove.ins:
%           tex natmove.dtx
%    (c) If you use LaTeX to generate files:
%           latex \let\install=y% \iffalse meta-comment
%
% Portions copyright (C) 1989-2003 by
%    Donald Arseneau
% Copyright (C) 2007-08 by
%    Joseph Wright <joseph.wright@morningstar2.co.uk>
%
% Part of this file is derived from cite.sty, to which the following
% license applies:
%--------------------------------------------------------------------
%     Copyright (C) 1989-2003 by Donald Arseneau
%     These macros may be freely transmitted, reproduced, or modified
%     provided that this notice is left intact.
%--------------------------------------------------------------------
%
% This work may be distributed and/or modified under the
% conditions of the LaTeX Project Public License, either
% version 1.3c of this license or (at your option) any later
% version. The latest version of this license is in
%    http://www.latex-project.org/lppl.txt
% and version 1.3c or later is part of all distributions of
% LaTeX version 2005/12/01 or later.
%
% This work has the LPPL maintenance status `maintained'.
%
% The current maintainer of this work is Joseph Wright.
%
% This work consists of the source file natmove.dtx
%                 and the derived files natmove.ins,
%                                       natmove.sty,
%                                       natmove.pdf,
%                                       natmove-manual.pdf and
%                                       jawltxdoc.sty
%
% TDS-ready files:
%    The compressed file achemso.tds.zip contains an unpacked version
%    of all of the files included here, and pre-compiled
%    documentation in PDF format.  Simply decompress achemso.tds.zip
%    in your local TeX directory, run your hash program (texhash,
%    initexmf --update-fndb, etc.) and everything will be ready to
%    go.  The user documentation for the package is called
%    natmove-manual.pdf; the file natmove.pdf includes the user
%    manual and the fully-indexed source code.
%
% Unpacking:
%    (a) If natmove.ins is present:
%           tex natmove.ins
%    (b) Without natmove.ins:
%           tex natmove.dtx
%    (c) If you use LaTeX to generate files:
%           latex \let\install=y\input{natmove.dtx}
%
% Documentation:
%    (a) Without write18 enabled:
%          pdflatex natmove.dtx
%          bibtex8 --wolfgang natmove
%          makeindex -s gind.ist natmove.idx
%          makeindex -s gglo.ist -o natmove.gls natmove.glo
%          pdflatex natmove.dtx
%          pdflatex natmove.dtx
%    (b) With write18 enabled:
%          pdflatex natmove.dtx
%          pdflatex natmove.dtx
%          pdflatex natmove.dtx
%
% Installation:
%     Copy natmove.sty to a location searched by TeX, and if required
%     by your TeX installation, run the appropriate command to build
%     a hash of files (texhash, initexmf --update-fndb, etc.)
%
% Note:
%     The jawltxdoc.sty file is not needed for installation,
%     only for building the documentation; it may be deleted
%     after producing the documentation (if necessary).
%
%<*ignore>
% This is all taken verbatim from Heiko Oberdiek's packages
\begingroup
  \def\x{LaTeX2e}%
\expandafter\endgroup
\ifcase 0\ifx\install y1\fi\expandafter
         \ifx\csname processbatchFile\endcsname\relax\else1\fi
         \ifx\fmtname\x\else 1\fi\relax
\else\csname fi\endcsname
%</ignore>
%<*install>
\input docstrip.tex
\keepsilent
\askforoverwritefalse
\preamble
 ----------------------------------------------------------------
 natmove --- Automatic citation moving with natbib
 Maintained by Joseph Wright
 E-mail: joseph.wright@morningstar2.co.uk
 Released under the LaTeX Project Public License v1.3c or later
 See http://www.latex-project.org/lppl.txt
 ----------------------------------------------------------------

\endpreamble
\Msg{Generating natmove files:}
\generate{\file{jawltxdoc.sty}{\from{\jobname.dtx}{jawltxdoc}}
}
\usedir{tex/latex/achemso}
\generate{\file{\jobname.sty}{\from{\jobname.dtx}{package}}
}
\usedir{source/latex/achemso}
\generate{\file{\jobname.ins}{\from{\jobname.dtx}{install}}
}
\endbatchfile
%</install>
%<*ignore>
\fi
% Will Robertson's trick
\immediate\write18{bibtex8 --wolfgang \jobname}
\immediate\write18{makeindex -s gind.ist -o \jobname.ind  \jobname.idx}
\immediate\write18{makeindex -s gglo.ist -o \jobname.gls  \jobname.glo}
%</ignore>
%<*driver>
\PassOptionsToClass{a4paper}{article}
\documentclass[german,english,UKenglish]{ltxdoc}
\EnableCrossrefs
\CodelineIndex
\RecordChanges
%\OnlyDescription
\usepackage[sort&compress,numbers,super]{natbib}
\usepackage{mciteplus}
\usepackage{jawltxdoc}
\begin{document}
  \DocInput{\jobname.dtx}
\end{document}
%</driver>
% \fi
%
%\CheckSum{158}
%
% \CharacterTable
%  {Upper-case    \A\B\C\D\E\F\G\H\I\J\K\L\M\N\O\P\Q\R\S\T\U\V\W\X\Y\Z
%   Lower-case    \a\b\c\d\e\f\g\h\i\j\k\l\m\n\o\p\q\r\s\t\u\v\w\x\y\z
%   Digits        \0\1\2\3\4\5\6\7\8\9
%   Exclamation   \!     Double quote  \"     Hash (number) \#
%   Dollar        \$     Percent       \%     Ampersand     \&
%   Acute accent  \'     Left paren    \(     Right paren   \)
%   Asterisk      \*     Plus          \+     Comma         \,
%   Minus         \-     Point         \.     Solidus       \/
%   Colon         \:     Semicolon     \;     Less than     \<
%   Equals        \=     Greater than  \>     Question mark \?
%   Commercial at \@     Left bracket  \[     Backslash     \\
%   Right bracket \]     Circumflex    \^     Underscore    \_
%   Grave accent  \`     Left brace    \{     Vertical bar  \|
%   Right brace   \}     Tilde         \~}
%
%\GetFileInfo{\jobname.sty}
%
%\DoNotIndex{\!,\.,\?,\@empty,\@firstoftwo,\@ifmtarg}
%\DoNotIndex{\@ifpackageloaded,\@m,\@nil,\@secondoftwo}
%\DoNotIndex{\AtBeginDocument,\begingroup,\catcode,\def,\else}
%\DoNotIndex{\endgroup,\expandafter,\fi,\frenchspacing,\futurelet}
%\DoNotIndex{\gdef,\global,\ifnum,\ifx,\ifNAT@super,\lastskip}
%\DoNotIndex{\leavevmode,\let,\long,\mathchardef,\NeedsTeXFormat}
%\DoNotIndex{,\newcommand,\newif,\PackageWarning,\ProvidesPackage,\Q}
%\DoNotIndex{\relax,\renewcommand,\RequirePackage,\sfcode,\skip@}
%\DoNotIndex{\space,\spacefactor,\unskip}
%
%\changes{v1.0}{2008/06/22}{First public release}
%
%\setkeys{lst}{language=[LaTeX]{TeX},moretexcs={citenum,
%  citeyear,citeauthor}}
%
%\title{\currpkg\ ---  Automatic citation moving with natbib^^A
%  \thanks{This file describes version \fileversion, last revised
%    \filedate.}}
%\author{Joseph Wright^^A
%  \thanks{E-mail: joseph.wright@morningstar2.co.uk}}
%\date{Released \filedate}
%
%\maketitle
%
%\newcommand*{\ACS}{\textsc{acs}}
%\begin{abstract}
% The \currpkg package adds the ability to move citations after
% punctuation automatically to the \pkg{natbib} package, in the same
% way as with \pkg{cite}.  With non-superscript citations, no action
% is taken.
%\end{abstract}
%
%\begin{multicols}{2}
%  \tableofcontents
%\end{multicols}
%
%\section{Using the package}
% The \currpkg package does only one job.  It brings the ability to
% move punctuation after citations, using code borrowed from the
% \pkg{cite} package.
%\begin{LaTeXexample}
%  Some text \cite{Coghill2006} some more text.\\
%  Some text ending a sentence \cite{Coghill2006}.
%\end{LaTeXexample}
% This is deactivated for other citation types.
%\begin{LaTeXexample}
%  Some text \citeyear{Coghill2006}.\\
%  Some text \citeauthor{Coghill2006}.\\
%  Some text \citenum{Coghill2006}.
%\end{LaTeXexample}
% The package does nothing if the \opt{super} option has not been
% given to \pkg{natbib}.  This means that the source can be written
% without needing to decide where citations will to appear, with the
% \opt{super} option for \pkg{natbib} controlling the result.
%
%\DescribeMacro{\natmovechars}
% One user macro is provided: \cs{natmovechars}.  This contains
% the characters which are moved before superscript punctuation.
% The default contents is |,;:.| and can be set using
% \cs{renewcommand*}:
%\begin{LaTeXexample}
%  \renewcommand*{\natmovechars}{.}
%  Some text \cite{Coghill2006},
%  more text \cite{Coghill2006}.
%\end{LaTeXexample}
%
%\StopEventually{%
%  \PrintChanges
%  \PrintIndex
%  \bibliographystyle{achemso}
%  \bibliography{achemso}}
%
%\iffalse
%<*package>
%\fi
%\section{The code}
%\begin{macro}{\nmv@id}
% The package file is designed to be usable with any document class.
% It sets up the basics, but leaves some settings to the class file.
%    \begin{macrocode}
\NeedsTeXFormat{LaTeX2e}
\def\nmv@id$#1: #2.#3 #4 #5-#6-#7 #8 #9${%
  #5/#6/#7\space v1.0\space}
\ProvidesPackage{natmove}
  [\nmv@id$Id: natmove.dtx 29 2008-08-22 07:47:26Z joseph $
   Automatic citation moving with natbib]
%    \end{macrocode}
%\end{macro}
%\begin{macro}{\nmv@ifmtarg}
%\begin{macro}{\nmv@xifmtarg}
% To keep down dependance on other packages, the very short code
% block from \pkg{ifmtarg} is copied here with an internal name.
%    \begin{macrocode}
\begingroup
  \catcode`\Q=3
  \long\gdef\nmv@ifmtarg#1{%
    \nmv@xifmtarg#1QQ\@secondoftwo\@firstoftwo\@nil}
  \long\gdef\nmv@xifmtarg#1#2Q#3#4#5\@nil{#4}
\endgroup
%    \end{macrocode}
%\end{macro}
%\end{macro}
%\begin{macro}{\ifnmv@cite}
% A flag is need to watch whether \cs{cite} or another macro is in
% use.
%    \begin{macrocode}
\newif\ifnmv@cite
%    \end{macrocode}
%\end{macro}
%\begin{macro}{\nmv@citex}
% Using the flag, either the new internal macro, or the
% \pkg{natbib} original, can be called.
%    \begin{macrocode}
\newcommand*{\nmv@citex}{%
  \ifnmv@cite
    \expandafter\nmv@citex@
  \else
    \expandafter\nmv@natcitex
  \fi}
%    \end{macrocode}
%\end{macro}
%\begin{macro}{\nmv@citex@}
% The new version of \cs{@citex} is needed that looks ahead of the
% citation using \cs{futurelet}.  There are three arguments to
% \cs{@citex} when using \pkg{natbib}.  Other than that, the trick
% used here is similar to that in \pkg{cite}.
%    \begin{macrocode}
\def\nmv@citex@[#1][#2]#3{%
  \leavevmode
  \skip@\lastskip
  \unskip
  \begingroup
%    \end{macrocode}
%\begin{macro}{\nmv@arg}
% The arguments of the macro now need to be saved, before handing off
% to the ``search'' macro.
%    \begin{macrocode}
    \def\nmv@arg{[#1][#2]{#3}}%
%    \end{macrocode}
%\end{macro}
%\begin{macro}{\nmv@citex@end}
% The entire block to be executed after punctuation is set up here,
% to make recursion easier.  The system used is very similar to that
% in \pkg{cite}.  Notice that the \cs{relax} is essential.
%    \begin{macrocode}
    \newcommand*{\nmv@citex@end}{%
        \expandafter\nmv@natcitex\nmv@arg
      \endgroup}%
    \global\nmv@citefalse
    \nmv@citex@getnext\relax}
%    \end{macrocode}
%\end{macro}
%\begin{macro}{\nmv@citex@getnext}
%\begin{macro}{\nmv@citex@next}
% The next token on the input stack is saved into
% \cs{nmv@citex@next}, after gobbling up one token.
%    \begin{macrocode}
\newcommand*{\nmv@citex@getnext}[1]{%
  \futurelet\nmv@citex@next\nmv@citex@punct}
%    \end{macrocode}
%\end{macro}
%\end{macro}
%\end{macro}
%\begin{macro}{\nmv@citex@punct}
%\begin{macro}{\nmv@citex@loop}
% The working macro for moving the punctuation.  This is very much
% like \cs{@citey} in the \pkg{cite} package. The initial assumption
% is that the loop will terminate, and so the recursion call will
% simply do the finalisation.
%    \begin{macrocode}
\newcommand*{\nmv@citex@punct}{%
  \let\nmv@citex@loop\nmv@citex@end
%    \end{macrocode}
% A check is made for doubled full stops.
%    \begin{macrocode}
  \ifx.\nmv@citex@next\@empty
    \ifnum\spacefactor<\nmv@citex@sfac\else
      \let\nmv@citex@next\relax
      \let\nmv@citex@loop\nmv@citex@getnext
    \fi
  \fi
%    \end{macrocode}
% The other cases are handled.
%    \begin{macrocode}
  \expandafter\nmv@citex@pnct\natmovechars\@empty
  \nmv@citex@loop}
%    \end{macrocode}
%\end{macro}
%\end{macro}
%\begin{macro}{\nmv@citex@pnct}
% The final part of the punctuation moving system.
%    \begin{macrocode}
\def\nmv@citex@pnct#1#2\@empty{%
  \ifx\nmv@citex@next#1\@empty
    #1%
    \let\nmv@citex@loop\nmv@citex@getnext
  \fi
  \ifx\@empty#2\@empty\else
    \expandafter\nmv@citex@pnct#2\@empty
  \fi}
%    \end{macrocode}
%\end{macro}
%\begin{macro}{\nmv@citex@sfac}
% The value of the spacing factor after a full stop is used to signal
% doubled punctuation.  For French spacing, a bit of patching is
% needed.
%    \begin{macrocode}
\mathchardef\nmv@citex@sfac3000
\expandafter\renewcommand\expandafter*\expandafter{\expandafter%
  \frenchspacing\expandafter}\expandafter{%
  \frenchspacing
  \mathchardef\nmv@citex@sfac1001
  \sfcode`\.\nmv@citex@sfac
  \sfcode`\?\nmv@citex@sfac
  \sfcode`\!\nmv@citex@sfac}
\ifnum\sfcode`\.=\@m
  \frenchspacing
\fi
%    \end{macrocode}
%\end{macro}
%\begin{macro}{\nmv@activate}
%\begin{macro}{\@citex}
%\begin{macro}{\nmv@natcitex}
%\begin{macro}{\cite}
%\begin{macro}{\nmv@cite}
%\begin{macro}{\nmv@notactivate}
% The swap has to be done at the beginning of the document. The
% internal flag from \pkg{natbib} is used, but under the
% circumstances we should be safe.  \cs{cite} is also patched to make
% the system active.  A
%    \begin{macrocode}
\newcommand*{\nmv@activate}{%
  \let\nmv@natcitex\@citex
  \let\@citex\nmv@citex
  \let\nmv@cite\cite
  \renewcommand*{\cite}[2][]{%
    \nmv@ifmtarg{##1}
      {\nmv@citetrue
       \nmv@cite{##2}}
      {\nmv@citefalse
       \nmv@cite[##1]{##2}}}}
\AtBeginDocument{
  \@ifpackageloaded{natbib}
    {\ifNAT@super
      \nmv@activate
     \else
      \nmv@notactivate
     \fi}
    {\PackageWarning{natmove}
       {The natbib package has not been loaded}}}
\newcommand*{\nmv@notactivate}{}
%    \end{macrocode}
%\end{macro}
%\end{macro}
%\end{macro}
%\end{macro}
%\end{macro}
%\end{macro}
%\begin{macro}{\natmovechars}
% A user macro is needed for moving characters.
%    \begin{macrocode}
\newcommand*{\natmovechars}{.,;:}
%    \end{macrocode}
%\end{macro}
%\iffalse
%</package>
%\fi
%
%\Finale
%
%\iffalse
%<*jawltxdoc>
\NeedsTeXFormat{LaTeX2e}
\ProvidesPackage{jawltxdoc}
\usepackage[T1]{fontenc}
\usepackage{lmodern}
\usepackage[final]{listings,graphicx,microtype}
\usepackage[scaled=0.95]{helvet}
\usepackage[version=3]{mhchem}
\usepackage[osf]{mathpazo}
\usepackage{booktabs,array,url,courier,xspace,varioref}
\usepackage{upgreek,ifpdf,float,caption,longtable,babel}
\begingroup
  \@ifundefined{eTeXversion}
    {\aftergroup\@gobble}
    {\aftergroup\@firstofone}
\endgroup
  {\usepackage{etoolbox}}
\floatstyle{plaintop}
\restylefloat{table}
\labelformat{figure}{\figurename~#1}
\labelformat{table}{\tablename~#1}
\ifpdf
  \usepackage{embedfile}
  \embedfile[%
    stringmethod=escape,%
    mimetype=plain/text,%
    desc={LaTeX docstrip source archive for package `\jobname'}%
    ]{\jobname.dtx}
\fi
\IfFileExists{\jobname.sty}
  {\usepackage{\jobname}}{}
\usepackage[numbered]{hypdoc}
\setcounter{IndexColumns}{2}
\newlength\LaTeXwidth
\newlength\LaTeXoutdent
\newlength\LaTeXgap
\setlength\LaTeXgap{1em}
\setlength\LaTeXoutdent{-0.15\textwidth}
\newbox\lst@samplebox
\edef\LaTeXexamplefile{\jobname.tmp}
\lst@RequireAspects{writefile}
\lstnewenvironment{LaTeXexample}[1][example]{%
  \global\let\lst@intname\@empty
  \ifcsname LaTeXcode#1\endcsname
    \expandafter\let\expandafter\LaTeXcode
      \csname LaTeXcode#1\endcsname
    \expandafter\let\expandafter\LaTeXcodeend
      \csname LaTeXcode#1end\endcsname
  \else
    \PackageError{jawltxdoc}
      {Undefined example type `#1'}
      \@ehd
    \let\LaTeXcode\relax
    \let\LaTeXcodeend\relax
  \fi
  \LaTeXcode}
  {\lst@EndWriteFile
   \LaTeXcodeend}
\newcommand*{\LaTeXcodeexample}{%
  \setbox\lst@samplebox=\hbox\bgroup
  \LaTeXcodefloat}
\let\LaTeXcoderesultonly\LaTeXcodeexample
\newcommand*{\LaTeXcodeexampleend}{%
  \egroup
  \setlength\LaTeXwidth{\wd\lst@samplebox}%
  \begin{list}{}{%
    \setlength\itemindent{0pt}
    \setlength\leftmargin\LaTeXoutdent
    \setlength\rightmargin{0pt}}%
    \item
      \setlength\LaTeXoutdent{-0.15\textwidth}
      \begin{minipage}[c]{%
        \textwidth-\LaTeXwidth-\LaTeXoutdent-\LaTeXgap}
        \LaTeXcodefloatend
      \end{minipage}%
      \hfill
      \begin{minipage}[c]{\LaTeXwidth}%
        \hbox to\linewidth{\box\lst@samplebox\hss}%
      \end{minipage}%
  \end{list}}
\newcommand*{\LaTeXcodefloat}{%
  \setkeys{lst}{tabsize=4,gobble=3,breakindent=0pt,
    basicstyle=\small\ttfamily,basewidth=0.51em,
    keywordstyle=\color{blue}}%
  \lst@BeginAlsoWriteFile{\LaTeXexamplefile}}
\let\LaTeXcodenoexample\LaTeXcodefloat
\let\LaTeXcodenoexampleend\@empty
\newcommand*{\LaTeXcodefloatend}{%
  \MakePercentComment\catcode`\^^M=10\relax
  \small
  {\setkeys{lst}{SelectCharTable=\lst@ReplaceInput{\^\^I}%
    {\lst@ProcessTabulator}}%
    \leavevmode \input{\LaTeXexamplefile}}%
  \MakePercentIgnore}
\newcommand*{\LaTeXcoderesultonlyend}{\egroup\LaTeXcodefloatend}
\lstnewenvironment{BibTeXexample}{%
  \global\let\lst@intname\@empty
  \setbox\lst@samplebox=\hbox\bgroup
  \setkeys{lst}{tabsize=4,gobble=3,breakindent=0pt,
    basicstyle=\small\ttfamily,basewidth=0.51em,
    keywordstyle=\color{black}}
  \lst@BeginAlsoWriteFile{\LaTeXexamplefile}}
 {\lst@EndWriteFile
   \LaTeXcodeexampleend}
\newcommand*{\DescribeOption}{%
  \leavevmode\@bsphack\begingroup\MakePrivateLetters
  \Describe@Option}
\newcommand*{\Describe@Option}[1]{\endgroup
              \marginpar{\raggedleft\PrintDescribeEnv{#1}}%
              \SpecialOptionIndex{#1}\@esphack\ignorespaces}
\newcommand*{\SpecialOptionIndex}[1]{\@bsphack
    \index{#1\actualchar{\protect\ttfamily#1}
           (option)\encapchar usage}%
    \index{options:\levelchar#1\actualchar{\protect\ttfamily#1}%
      \encapchar usage}\@esphack}
\newcommand*{\indexopt}[1]{\DescribeOption{#1}\opt{#1}}
\newcommand*{\DescribeOptionInfo}[2]{%
  \DescribeOption{#1}%
  \opt{#1=\meta{#2}}\xspace}
\newcommand*{\ofixarg}[1]{%
  {\ttfamily[}%
  \ifmmode \expandafter \nfss@text \fi
  {%
    \meta@font@select
    \edef\meta@hyphen@restore{%
      \hyphenchar\the\font\the\hyphenchar\font}%
    \hyphenchar\font\m@ne
    \language\l@nohyphenation
    #1\/%
    \meta@hyphen@restore
    }%
    {\ttfamily]}}
\newcommand*{\pkg}[1]{\textsf{#1}}
\newcommand*{\currpkg}{\pkg{\jobname}\xspace}
\newcommand*{\opt}[1]{\texttt{#1}}
\newcommand*{\defaultopt}[1]{\opt{\textbf{#1}}}
\newcommand*{\file}[1]{\texttt{#1}}
\newcommand*{\ext}[1]{\file{.#1}}
\newcommand*{\latin}[1]{\emph{#1}}
\newcommand*{\etc}{%
  \@ifnextchar.
    {\latin{etc}}
    {\latin{etc}.\xspace}}
\newcommand*{\eg}{%
  \@ifnextchar.
    {\latin{e.g}}
    {\latin{e.g}.\xspace}}
\newcommand*{\ie}{%
  \@ifnextchar.
    {\latin{i.e}}
    {\latin{i.e}.\xspace}}
\newcommand*{\etal}{%
  \@ifnextchar.
    {\latin{et~al.}}
    {\latin{et~al}.\xspace}}
\newcommand*{\AMS}{{\protect\usefont{OMS}{cmsy}{m}{n}%
  A\kern-.1667em\lower.5ex\hbox{M}\kern-.125emS}}
\providecommand*{\eTeX}{\ensuremath{\varepsilon}-\TeX}
\DeclareRobustCommand*{\XeTeX}
  {X\kern-.125em\lower.5ex\hbox{\reflectbox{E}}\kern-.1667em\TeX}
\providecommand*{\CTAN}{\textsc{ctan}}
\@ifpackageloaded{etoolbox}
  {\patchcmd{\@addmarginpar}
    {\@latex@warning@no@line {Marginpar on page \thepage\space moved}}
    {\relax}{}{}}
  {}
\newcounter{argument}
\g@addto@macro\endmacro{\setcounter{argument}{0}}
\newcommand*\darg[1]{%
  \stepcounter{argument}%
  {\ttfamily\char`\#\theargument~:~}#1\par\noindent\ignorespaces}
\newcommand*\doarg[1]{%
  \stepcounter{argument}%
  {\ttfamily\makebox[0pt][r]{[}%
   \char`\#\theargument]:~}#1\par\noindent\ignorespaces}
%</jawltxdoc>
%\fi

%
% Documentation:
%    (a) Without write18 enabled:
%          pdflatex natmove.dtx
%          bibtex8 --wolfgang natmove
%          makeindex -s gind.ist natmove.idx
%          makeindex -s gglo.ist -o natmove.gls natmove.glo
%          pdflatex natmove.dtx
%          pdflatex natmove.dtx
%    (b) With write18 enabled:
%          pdflatex natmove.dtx
%          pdflatex natmove.dtx
%          pdflatex natmove.dtx
%
% Installation:
%     Copy natmove.sty to a location searched by TeX, and if required
%     by your TeX installation, run the appropriate command to build
%     a hash of files (texhash, initexmf --update-fndb, etc.)
%
% Note:
%     The jawltxdoc.sty file is not needed for installation,
%     only for building the documentation; it may be deleted
%     after producing the documentation (if necessary).
%
%<*ignore>
% This is all taken verbatim from Heiko Oberdiek's packages
\begingroup
  \def\x{LaTeX2e}%
\expandafter\endgroup
\ifcase 0\ifx\install y1\fi\expandafter
         \ifx\csname processbatchFile\endcsname\relax\else1\fi
         \ifx\fmtname\x\else 1\fi\relax
\else\csname fi\endcsname
%</ignore>
%<*install>
\input docstrip.tex
\keepsilent
\askforoverwritefalse
\preamble
 ----------------------------------------------------------------
 natmove --- Automatic citation moving with natbib
 Maintained by Joseph Wright
 E-mail: joseph.wright@morningstar2.co.uk
 Released under the LaTeX Project Public License v1.3c or later
 See http://www.latex-project.org/lppl.txt
 ----------------------------------------------------------------

\endpreamble
\Msg{Generating natmove files:}
\generate{\file{jawltxdoc.sty}{\from{\jobname.dtx}{jawltxdoc}}
}
\usedir{tex/latex/achemso}
\generate{\file{\jobname.sty}{\from{\jobname.dtx}{package}}
}
\usedir{source/latex/achemso}
\generate{\file{\jobname.ins}{\from{\jobname.dtx}{install}}
}
\endbatchfile
%</install>
%<*ignore>
\fi
% Will Robertson's trick
\immediate\write18{bibtex8 --wolfgang \jobname}
\immediate\write18{makeindex -s gind.ist -o \jobname.ind  \jobname.idx}
\immediate\write18{makeindex -s gglo.ist -o \jobname.gls  \jobname.glo}
%</ignore>
%<*driver>
\PassOptionsToClass{a4paper}{article}
\documentclass[german,english,UKenglish]{ltxdoc}
\EnableCrossrefs
\CodelineIndex
\RecordChanges
%\OnlyDescription
\usepackage[sort&compress,numbers,super]{natbib}
\usepackage{mciteplus}
\usepackage{jawltxdoc}
\begin{document}
  \DocInput{\jobname.dtx}
\end{document}
%</driver>
% \fi
%
%\CheckSum{158}
%
% \CharacterTable
%  {Upper-case    \A\B\C\D\E\F\G\H\I\J\K\L\M\N\O\P\Q\R\S\T\U\V\W\X\Y\Z
%   Lower-case    \a\b\c\d\e\f\g\h\i\j\k\l\m\n\o\p\q\r\s\t\u\v\w\x\y\z
%   Digits        \0\1\2\3\4\5\6\7\8\9
%   Exclamation   \!     Double quote  \"     Hash (number) \#
%   Dollar        \$     Percent       \%     Ampersand     \&
%   Acute accent  \'     Left paren    \(     Right paren   \)
%   Asterisk      \*     Plus          \+     Comma         \,
%   Minus         \-     Point         \.     Solidus       \/
%   Colon         \:     Semicolon     \;     Less than     \<
%   Equals        \=     Greater than  \>     Question mark \?
%   Commercial at \@     Left bracket  \[     Backslash     \\
%   Right bracket \]     Circumflex    \^     Underscore    \_
%   Grave accent  \`     Left brace    \{     Vertical bar  \|
%   Right brace   \}     Tilde         \~}
%
%\GetFileInfo{\jobname.sty}
%
%\DoNotIndex{\!,\.,\?,\@empty,\@firstoftwo,\@ifmtarg}
%\DoNotIndex{\@ifpackageloaded,\@m,\@nil,\@secondoftwo}
%\DoNotIndex{\AtBeginDocument,\begingroup,\catcode,\def,\else}
%\DoNotIndex{\endgroup,\expandafter,\fi,\frenchspacing,\futurelet}
%\DoNotIndex{\gdef,\global,\ifnum,\ifx,\ifNAT@super,\lastskip}
%\DoNotIndex{\leavevmode,\let,\long,\mathchardef,\NeedsTeXFormat}
%\DoNotIndex{,\newcommand,\newif,\PackageWarning,\ProvidesPackage,\Q}
%\DoNotIndex{\relax,\renewcommand,\RequirePackage,\sfcode,\skip@}
%\DoNotIndex{\space,\spacefactor,\unskip}
%
%\changes{v1.0}{2008/06/22}{First public release}
%
%\setkeys{lst}{language=[LaTeX]{TeX},moretexcs={citenum,
%  citeyear,citeauthor}}
%
%\title{\currpkg\ ---  Automatic citation moving with natbib^^A
%  \thanks{This file describes version \fileversion, last revised
%    \filedate.}}
%\author{Joseph Wright^^A
%  \thanks{E-mail: joseph.wright@morningstar2.co.uk}}
%\date{Released \filedate}
%
%\maketitle
%
%\newcommand*{\ACS}{\textsc{acs}}
%\begin{abstract}
% The \currpkg package adds the ability to move citations after
% punctuation automatically to the \pkg{natbib} package, in the same
% way as with \pkg{cite}.  With non-superscript citations, no action
% is taken.
%\end{abstract}
%
%\begin{multicols}{2}
%  \tableofcontents
%\end{multicols}
%
%\section{Using the package}
% The \currpkg package does only one job.  It brings the ability to
% move punctuation after citations, using code borrowed from the
% \pkg{cite} package.
%\begin{LaTeXexample}
%  Some text \cite{Coghill2006} some more text.\\
%  Some text ending a sentence \cite{Coghill2006}.
%\end{LaTeXexample}
% This is deactivated for other citation types.
%\begin{LaTeXexample}
%  Some text \citeyear{Coghill2006}.\\
%  Some text \citeauthor{Coghill2006}.\\
%  Some text \citenum{Coghill2006}.
%\end{LaTeXexample}
% The package does nothing if the \opt{super} option has not been
% given to \pkg{natbib}.  This means that the source can be written
% without needing to decide where citations will to appear, with the
% \opt{super} option for \pkg{natbib} controlling the result.
%
%\DescribeMacro{\natmovechars}
% One user macro is provided: \cs{natmovechars}.  This contains
% the characters which are moved before superscript punctuation.
% The default contents is |,;:.| and can be set using
% \cs{renewcommand*}:
%\begin{LaTeXexample}
%  \renewcommand*{\natmovechars}{.}
%  Some text \cite{Coghill2006},
%  more text \cite{Coghill2006}.
%\end{LaTeXexample}
%
%\StopEventually{%
%  \PrintChanges
%  \PrintIndex
%  \bibliographystyle{achemso}
%  \bibliography{achemso}}
%
%\iffalse
%<*package>
%\fi
%\section{The code}
%\begin{macro}{\nmv@id}
% The package file is designed to be usable with any document class.
% It sets up the basics, but leaves some settings to the class file.
%    \begin{macrocode}
\NeedsTeXFormat{LaTeX2e}
\def\nmv@id$#1: #2.#3 #4 #5-#6-#7 #8 #9${%
  #5/#6/#7\space v1.0\space}
\ProvidesPackage{natmove}
  [\nmv@id$Id: natmove.dtx 29 2008-08-22 07:47:26Z joseph $
   Automatic citation moving with natbib]
%    \end{macrocode}
%\end{macro}
%\begin{macro}{\nmv@ifmtarg}
%\begin{macro}{\nmv@xifmtarg}
% To keep down dependance on other packages, the very short code
% block from \pkg{ifmtarg} is copied here with an internal name.
%    \begin{macrocode}
\begingroup
  \catcode`\Q=3
  \long\gdef\nmv@ifmtarg#1{%
    \nmv@xifmtarg#1QQ\@secondoftwo\@firstoftwo\@nil}
  \long\gdef\nmv@xifmtarg#1#2Q#3#4#5\@nil{#4}
\endgroup
%    \end{macrocode}
%\end{macro}
%\end{macro}
%\begin{macro}{\ifnmv@cite}
% A flag is need to watch whether \cs{cite} or another macro is in
% use.
%    \begin{macrocode}
\newif\ifnmv@cite
%    \end{macrocode}
%\end{macro}
%\begin{macro}{\nmv@citex}
% Using the flag, either the new internal macro, or the
% \pkg{natbib} original, can be called.
%    \begin{macrocode}
\newcommand*{\nmv@citex}{%
  \ifnmv@cite
    \expandafter\nmv@citex@
  \else
    \expandafter\nmv@natcitex
  \fi}
%    \end{macrocode}
%\end{macro}
%\begin{macro}{\nmv@citex@}
% The new version of \cs{@citex} is needed that looks ahead of the
% citation using \cs{futurelet}.  There are three arguments to
% \cs{@citex} when using \pkg{natbib}.  Other than that, the trick
% used here is similar to that in \pkg{cite}.
%    \begin{macrocode}
\def\nmv@citex@[#1][#2]#3{%
  \leavevmode
  \skip@\lastskip
  \unskip
  \begingroup
%    \end{macrocode}
%\begin{macro}{\nmv@arg}
% The arguments of the macro now need to be saved, before handing off
% to the ``search'' macro.
%    \begin{macrocode}
    \def\nmv@arg{[#1][#2]{#3}}%
%    \end{macrocode}
%\end{macro}
%\begin{macro}{\nmv@citex@end}
% The entire block to be executed after punctuation is set up here,
% to make recursion easier.  The system used is very similar to that
% in \pkg{cite}.  Notice that the \cs{relax} is essential.
%    \begin{macrocode}
    \newcommand*{\nmv@citex@end}{%
        \expandafter\nmv@natcitex\nmv@arg
      \endgroup}%
    \global\nmv@citefalse
    \nmv@citex@getnext\relax}
%    \end{macrocode}
%\end{macro}
%\begin{macro}{\nmv@citex@getnext}
%\begin{macro}{\nmv@citex@next}
% The next token on the input stack is saved into
% \cs{nmv@citex@next}, after gobbling up one token.
%    \begin{macrocode}
\newcommand*{\nmv@citex@getnext}[1]{%
  \futurelet\nmv@citex@next\nmv@citex@punct}
%    \end{macrocode}
%\end{macro}
%\end{macro}
%\end{macro}
%\begin{macro}{\nmv@citex@punct}
%\begin{macro}{\nmv@citex@loop}
% The working macro for moving the punctuation.  This is very much
% like \cs{@citey} in the \pkg{cite} package. The initial assumption
% is that the loop will terminate, and so the recursion call will
% simply do the finalisation.
%    \begin{macrocode}
\newcommand*{\nmv@citex@punct}{%
  \let\nmv@citex@loop\nmv@citex@end
%    \end{macrocode}
% A check is made for doubled full stops.
%    \begin{macrocode}
  \ifx.\nmv@citex@next\@empty
    \ifnum\spacefactor<\nmv@citex@sfac\else
      \let\nmv@citex@next\relax
      \let\nmv@citex@loop\nmv@citex@getnext
    \fi
  \fi
%    \end{macrocode}
% The other cases are handled.
%    \begin{macrocode}
  \expandafter\nmv@citex@pnct\natmovechars\@empty
  \nmv@citex@loop}
%    \end{macrocode}
%\end{macro}
%\end{macro}
%\begin{macro}{\nmv@citex@pnct}
% The final part of the punctuation moving system.
%    \begin{macrocode}
\def\nmv@citex@pnct#1#2\@empty{%
  \ifx\nmv@citex@next#1\@empty
    #1%
    \let\nmv@citex@loop\nmv@citex@getnext
  \fi
  \ifx\@empty#2\@empty\else
    \expandafter\nmv@citex@pnct#2\@empty
  \fi}
%    \end{macrocode}
%\end{macro}
%\begin{macro}{\nmv@citex@sfac}
% The value of the spacing factor after a full stop is used to signal
% doubled punctuation.  For French spacing, a bit of patching is
% needed.
%    \begin{macrocode}
\mathchardef\nmv@citex@sfac3000
\expandafter\renewcommand\expandafter*\expandafter{\expandafter%
  \frenchspacing\expandafter}\expandafter{%
  \frenchspacing
  \mathchardef\nmv@citex@sfac1001
  \sfcode`\.\nmv@citex@sfac
  \sfcode`\?\nmv@citex@sfac
  \sfcode`\!\nmv@citex@sfac}
\ifnum\sfcode`\.=\@m
  \frenchspacing
\fi
%    \end{macrocode}
%\end{macro}
%\begin{macro}{\nmv@activate}
%\begin{macro}{\@citex}
%\begin{macro}{\nmv@natcitex}
%\begin{macro}{\cite}
%\begin{macro}{\nmv@cite}
%\begin{macro}{\nmv@notactivate}
% The swap has to be done at the beginning of the document. The
% internal flag from \pkg{natbib} is used, but under the
% circumstances we should be safe.  \cs{cite} is also patched to make
% the system active.  A
%    \begin{macrocode}
\newcommand*{\nmv@activate}{%
  \let\nmv@natcitex\@citex
  \let\@citex\nmv@citex
  \let\nmv@cite\cite
  \renewcommand*{\cite}[2][]{%
    \nmv@ifmtarg{##1}
      {\nmv@citetrue
       \nmv@cite{##2}}
      {\nmv@citefalse
       \nmv@cite[##1]{##2}}}}
\AtBeginDocument{
  \@ifpackageloaded{natbib}
    {\ifNAT@super
      \nmv@activate
     \else
      \nmv@notactivate
     \fi}
    {\PackageWarning{natmove}
       {The natbib package has not been loaded}}}
\newcommand*{\nmv@notactivate}{}
%    \end{macrocode}
%\end{macro}
%\end{macro}
%\end{macro}
%\end{macro}
%\end{macro}
%\end{macro}
%\begin{macro}{\natmovechars}
% A user macro is needed for moving characters.
%    \begin{macrocode}
\newcommand*{\natmovechars}{.,;:}
%    \end{macrocode}
%\end{macro}
%\iffalse
%</package>
%\fi
%
%\Finale
%
%\iffalse
%<*jawltxdoc>
\NeedsTeXFormat{LaTeX2e}
\ProvidesPackage{jawltxdoc}
\usepackage[T1]{fontenc}
\usepackage{lmodern}
\usepackage[final]{listings,graphicx,microtype}
\usepackage[scaled=0.95]{helvet}
\usepackage[version=3]{mhchem}
\usepackage[osf]{mathpazo}
\usepackage{booktabs,array,url,courier,xspace,varioref}
\usepackage{upgreek,ifpdf,float,caption,longtable,babel}
\begingroup
  \@ifundefined{eTeXversion}
    {\aftergroup\@gobble}
    {\aftergroup\@firstofone}
\endgroup
  {\usepackage{etoolbox}}
\floatstyle{plaintop}
\restylefloat{table}
\labelformat{figure}{\figurename~#1}
\labelformat{table}{\tablename~#1}
\ifpdf
  \usepackage{embedfile}
  \embedfile[%
    stringmethod=escape,%
    mimetype=plain/text,%
    desc={LaTeX docstrip source archive for package `\jobname'}%
    ]{\jobname.dtx}
\fi
\IfFileExists{\jobname.sty}
  {\usepackage{\jobname}}{}
\usepackage[numbered]{hypdoc}
\setcounter{IndexColumns}{2}
\newlength\LaTeXwidth
\newlength\LaTeXoutdent
\newlength\LaTeXgap
\setlength\LaTeXgap{1em}
\setlength\LaTeXoutdent{-0.15\textwidth}
\newbox\lst@samplebox
\edef\LaTeXexamplefile{\jobname.tmp}
\lst@RequireAspects{writefile}
\lstnewenvironment{LaTeXexample}[1][example]{%
  \global\let\lst@intname\@empty
  \ifcsname LaTeXcode#1\endcsname
    \expandafter\let\expandafter\LaTeXcode
      \csname LaTeXcode#1\endcsname
    \expandafter\let\expandafter\LaTeXcodeend
      \csname LaTeXcode#1end\endcsname
  \else
    \PackageError{jawltxdoc}
      {Undefined example type `#1'}
      \@ehd
    \let\LaTeXcode\relax
    \let\LaTeXcodeend\relax
  \fi
  \LaTeXcode}
  {\lst@EndWriteFile
   \LaTeXcodeend}
\newcommand*{\LaTeXcodeexample}{%
  \setbox\lst@samplebox=\hbox\bgroup
  \LaTeXcodefloat}
\let\LaTeXcoderesultonly\LaTeXcodeexample
\newcommand*{\LaTeXcodeexampleend}{%
  \egroup
  \setlength\LaTeXwidth{\wd\lst@samplebox}%
  \begin{list}{}{%
    \setlength\itemindent{0pt}
    \setlength\leftmargin\LaTeXoutdent
    \setlength\rightmargin{0pt}}%
    \item
      \setlength\LaTeXoutdent{-0.15\textwidth}
      \begin{minipage}[c]{%
        \textwidth-\LaTeXwidth-\LaTeXoutdent-\LaTeXgap}
        \LaTeXcodefloatend
      \end{minipage}%
      \hfill
      \begin{minipage}[c]{\LaTeXwidth}%
        \hbox to\linewidth{\box\lst@samplebox\hss}%
      \end{minipage}%
  \end{list}}
\newcommand*{\LaTeXcodefloat}{%
  \setkeys{lst}{tabsize=4,gobble=3,breakindent=0pt,
    basicstyle=\small\ttfamily,basewidth=0.51em,
    keywordstyle=\color{blue}}%
  \lst@BeginAlsoWriteFile{\LaTeXexamplefile}}
\let\LaTeXcodenoexample\LaTeXcodefloat
\let\LaTeXcodenoexampleend\@empty
\newcommand*{\LaTeXcodefloatend}{%
  \MakePercentComment\catcode`\^^M=10\relax
  \small
  {\setkeys{lst}{SelectCharTable=\lst@ReplaceInput{\^\^I}%
    {\lst@ProcessTabulator}}%
    \leavevmode \input{\LaTeXexamplefile}}%
  \MakePercentIgnore}
\newcommand*{\LaTeXcoderesultonlyend}{\egroup\LaTeXcodefloatend}
\lstnewenvironment{BibTeXexample}{%
  \global\let\lst@intname\@empty
  \setbox\lst@samplebox=\hbox\bgroup
  \setkeys{lst}{tabsize=4,gobble=3,breakindent=0pt,
    basicstyle=\small\ttfamily,basewidth=0.51em,
    keywordstyle=\color{black}}
  \lst@BeginAlsoWriteFile{\LaTeXexamplefile}}
 {\lst@EndWriteFile
   \LaTeXcodeexampleend}
\newcommand*{\DescribeOption}{%
  \leavevmode\@bsphack\begingroup\MakePrivateLetters
  \Describe@Option}
\newcommand*{\Describe@Option}[1]{\endgroup
              \marginpar{\raggedleft\PrintDescribeEnv{#1}}%
              \SpecialOptionIndex{#1}\@esphack\ignorespaces}
\newcommand*{\SpecialOptionIndex}[1]{\@bsphack
    \index{#1\actualchar{\protect\ttfamily#1}
           (option)\encapchar usage}%
    \index{options:\levelchar#1\actualchar{\protect\ttfamily#1}%
      \encapchar usage}\@esphack}
\newcommand*{\indexopt}[1]{\DescribeOption{#1}\opt{#1}}
\newcommand*{\DescribeOptionInfo}[2]{%
  \DescribeOption{#1}%
  \opt{#1=\meta{#2}}\xspace}
\newcommand*{\ofixarg}[1]{%
  {\ttfamily[}%
  \ifmmode \expandafter \nfss@text \fi
  {%
    \meta@font@select
    \edef\meta@hyphen@restore{%
      \hyphenchar\the\font\the\hyphenchar\font}%
    \hyphenchar\font\m@ne
    \language\l@nohyphenation
    #1\/%
    \meta@hyphen@restore
    }%
    {\ttfamily]}}
\newcommand*{\pkg}[1]{\textsf{#1}}
\newcommand*{\currpkg}{\pkg{\jobname}\xspace}
\newcommand*{\opt}[1]{\texttt{#1}}
\newcommand*{\defaultopt}[1]{\opt{\textbf{#1}}}
\newcommand*{\file}[1]{\texttt{#1}}
\newcommand*{\ext}[1]{\file{.#1}}
\newcommand*{\latin}[1]{\emph{#1}}
\newcommand*{\etc}{%
  \@ifnextchar.
    {\latin{etc}}
    {\latin{etc}.\xspace}}
\newcommand*{\eg}{%
  \@ifnextchar.
    {\latin{e.g}}
    {\latin{e.g}.\xspace}}
\newcommand*{\ie}{%
  \@ifnextchar.
    {\latin{i.e}}
    {\latin{i.e}.\xspace}}
\newcommand*{\etal}{%
  \@ifnextchar.
    {\latin{et~al.}}
    {\latin{et~al}.\xspace}}
\newcommand*{\AMS}{{\protect\usefont{OMS}{cmsy}{m}{n}%
  A\kern-.1667em\lower.5ex\hbox{M}\kern-.125emS}}
\providecommand*{\eTeX}{\ensuremath{\varepsilon}-\TeX}
\DeclareRobustCommand*{\XeTeX}
  {X\kern-.125em\lower.5ex\hbox{\reflectbox{E}}\kern-.1667em\TeX}
\providecommand*{\CTAN}{\textsc{ctan}}
\@ifpackageloaded{etoolbox}
  {\patchcmd{\@addmarginpar}
    {\@latex@warning@no@line {Marginpar on page \thepage\space moved}}
    {\relax}{}{}}
  {}
\newcounter{argument}
\g@addto@macro\endmacro{\setcounter{argument}{0}}
\newcommand*\darg[1]{%
  \stepcounter{argument}%
  {\ttfamily\char`\#\theargument~:~}#1\par\noindent\ignorespaces}
\newcommand*\doarg[1]{%
  \stepcounter{argument}%
  {\ttfamily\makebox[0pt][r]{[}%
   \char`\#\theargument]:~}#1\par\noindent\ignorespaces}
%</jawltxdoc>
%\fi

%
% Documentation:
%    (a) Without write18 enabled:
%          pdflatex natmove.dtx
%          bibtex8 --wolfgang natmove
%          makeindex -s gind.ist natmove.idx
%          makeindex -s gglo.ist -o natmove.gls natmove.glo
%          pdflatex natmove.dtx
%          pdflatex natmove.dtx
%    (b) With write18 enabled:
%          pdflatex natmove.dtx
%          pdflatex natmove.dtx
%          pdflatex natmove.dtx
%
% Installation:
%     Copy natmove.sty to a location searched by TeX, and if required
%     by your TeX installation, run the appropriate command to build
%     a hash of files (texhash, initexmf --update-fndb, etc.)
%
% Note:
%     The jawltxdoc.sty file is not needed for installation,
%     only for building the documentation; it may be deleted
%     after producing the documentation (if necessary).
%
%<*ignore>
% This is all taken verbatim from Heiko Oberdiek's packages
\begingroup
  \def\x{LaTeX2e}%
\expandafter\endgroup
\ifcase 0\ifx\install y1\fi\expandafter
         \ifx\csname processbatchFile\endcsname\relax\else1\fi
         \ifx\fmtname\x\else 1\fi\relax
\else\csname fi\endcsname
%</ignore>
%<*install>
\input docstrip.tex
\keepsilent
\askforoverwritefalse
\preamble
 ----------------------------------------------------------------
 natmove --- Automatic citation moving with natbib
 Maintained by Joseph Wright
 E-mail: joseph.wright@morningstar2.co.uk
 Released under the LaTeX Project Public License v1.3c or later
 See http://www.latex-project.org/lppl.txt
 ----------------------------------------------------------------

\endpreamble
\Msg{Generating natmove files:}
\generate{\file{jawltxdoc.sty}{\from{\jobname.dtx}{jawltxdoc}}
}
\usedir{tex/latex/achemso}
\generate{\file{\jobname.sty}{\from{\jobname.dtx}{package}}
}
\usedir{source/latex/achemso}
\generate{\file{\jobname.ins}{\from{\jobname.dtx}{install}}
}
\endbatchfile
%</install>
%<*ignore>
\fi
% Will Robertson's trick
\immediate\write18{bibtex8 --wolfgang \jobname}
\immediate\write18{makeindex -s gind.ist -o \jobname.ind  \jobname.idx}
\immediate\write18{makeindex -s gglo.ist -o \jobname.gls  \jobname.glo}
%</ignore>
%<*driver>
\PassOptionsToClass{a4paper}{article}
\documentclass[german,english,UKenglish]{ltxdoc}
\EnableCrossrefs
\CodelineIndex
\RecordChanges
%\OnlyDescription
\usepackage[sort&compress,numbers,super]{natbib}
\usepackage{mciteplus}
\usepackage{jawltxdoc}
\begin{document}
  \DocInput{\jobname.dtx}
\end{document}
%</driver>
% \fi
%
%\CheckSum{158}
%
% \CharacterTable
%  {Upper-case    \A\B\C\D\E\F\G\H\I\J\K\L\M\N\O\P\Q\R\S\T\U\V\W\X\Y\Z
%   Lower-case    \a\b\c\d\e\f\g\h\i\j\k\l\m\n\o\p\q\r\s\t\u\v\w\x\y\z
%   Digits        \0\1\2\3\4\5\6\7\8\9
%   Exclamation   \!     Double quote  \"     Hash (number) \#
%   Dollar        \$     Percent       \%     Ampersand     \&
%   Acute accent  \'     Left paren    \(     Right paren   \)
%   Asterisk      \*     Plus          \+     Comma         \,
%   Minus         \-     Point         \.     Solidus       \/
%   Colon         \:     Semicolon     \;     Less than     \<
%   Equals        \=     Greater than  \>     Question mark \?
%   Commercial at \@     Left bracket  \[     Backslash     \\
%   Right bracket \]     Circumflex    \^     Underscore    \_
%   Grave accent  \`     Left brace    \{     Vertical bar  \|
%   Right brace   \}     Tilde         \~}
%
%\GetFileInfo{\jobname.sty}
%
%\DoNotIndex{\!,\.,\?,\@empty,\@firstoftwo,\@ifmtarg}
%\DoNotIndex{\@ifpackageloaded,\@m,\@nil,\@secondoftwo}
%\DoNotIndex{\AtBeginDocument,\begingroup,\catcode,\def,\else}
%\DoNotIndex{\endgroup,\expandafter,\fi,\frenchspacing,\futurelet}
%\DoNotIndex{\gdef,\global,\ifnum,\ifx,\ifNAT@super,\lastskip}
%\DoNotIndex{\leavevmode,\let,\long,\mathchardef,\NeedsTeXFormat}
%\DoNotIndex{,\newcommand,\newif,\PackageWarning,\ProvidesPackage,\Q}
%\DoNotIndex{\relax,\renewcommand,\RequirePackage,\sfcode,\skip@}
%\DoNotIndex{\space,\spacefactor,\unskip}
%
%\changes{v1.0}{2008/06/22}{First public release}
%
%\setkeys{lst}{language=[LaTeX]{TeX},moretexcs={citenum,
%  citeyear,citeauthor}}
%
%\title{\currpkg\ ---  Automatic citation moving with natbib^^A
%  \thanks{This file describes version \fileversion, last revised
%    \filedate.}}
%\author{Joseph Wright^^A
%  \thanks{E-mail: joseph.wright@morningstar2.co.uk}}
%\date{Released \filedate}
%
%\maketitle
%
%\newcommand*{\ACS}{\textsc{acs}}
%\begin{abstract}
% The \currpkg package adds the ability to move citations after
% punctuation automatically to the \pkg{natbib} package, in the same
% way as with \pkg{cite}.  With non-superscript citations, no action
% is taken.
%\end{abstract}
%
%\begin{multicols}{2}
%  \tableofcontents
%\end{multicols}
%
%\section{Using the package}
% The \currpkg package does only one job.  It brings the ability to
% move punctuation after citations, using code borrowed from the
% \pkg{cite} package.
%\begin{LaTeXexample}
%  Some text \cite{Coghill2006} some more text.\\
%  Some text ending a sentence \cite{Coghill2006}.
%\end{LaTeXexample}
% This is deactivated for other citation types.
%\begin{LaTeXexample}
%  Some text \citeyear{Coghill2006}.\\
%  Some text \citeauthor{Coghill2006}.\\
%  Some text \citenum{Coghill2006}.
%\end{LaTeXexample}
% The package does nothing if the \opt{super} option has not been
% given to \pkg{natbib}.  This means that the source can be written
% without needing to decide where citations will to appear, with the
% \opt{super} option for \pkg{natbib} controlling the result.
%
%\DescribeMacro{\natmovechars}
% One user macro is provided: \cs{natmovechars}.  This contains
% the characters which are moved before superscript punctuation.
% The default contents is |,;:.| and can be set using
% \cs{renewcommand*}:
%\begin{LaTeXexample}
%  \renewcommand*{\natmovechars}{.}
%  Some text \cite{Coghill2006},
%  more text \cite{Coghill2006}.
%\end{LaTeXexample}
%
%\StopEventually{%
%  \PrintChanges
%  \PrintIndex
%  \bibliographystyle{achemso}
%  \bibliography{achemso}}
%
%\iffalse
%<*package>
%\fi
%\section{The code}
%\begin{macro}{\nmv@id}
% The package file is designed to be usable with any document class.
% It sets up the basics, but leaves some settings to the class file.
%    \begin{macrocode}
\NeedsTeXFormat{LaTeX2e}
\def\nmv@id$#1: #2.#3 #4 #5-#6-#7 #8 #9${%
  #5/#6/#7\space v1.0\space}
\ProvidesPackage{natmove}
  [\nmv@id$Id: natmove.dtx 29 2008-08-22 07:47:26Z joseph $
   Automatic citation moving with natbib]
%    \end{macrocode}
%\end{macro}
%\begin{macro}{\nmv@ifmtarg}
%\begin{macro}{\nmv@xifmtarg}
% To keep down dependance on other packages, the very short code
% block from \pkg{ifmtarg} is copied here with an internal name.
%    \begin{macrocode}
\begingroup
  \catcode`\Q=3
  \long\gdef\nmv@ifmtarg#1{%
    \nmv@xifmtarg#1QQ\@secondoftwo\@firstoftwo\@nil}
  \long\gdef\nmv@xifmtarg#1#2Q#3#4#5\@nil{#4}
\endgroup
%    \end{macrocode}
%\end{macro}
%\end{macro}
%\begin{macro}{\ifnmv@cite}
% A flag is need to watch whether \cs{cite} or another macro is in
% use.
%    \begin{macrocode}
\newif\ifnmv@cite
%    \end{macrocode}
%\end{macro}
%\begin{macro}{\nmv@citex}
% Using the flag, either the new internal macro, or the
% \pkg{natbib} original, can be called.
%    \begin{macrocode}
\newcommand*{\nmv@citex}{%
  \ifnmv@cite
    \expandafter\nmv@citex@
  \else
    \expandafter\nmv@natcitex
  \fi}
%    \end{macrocode}
%\end{macro}
%\begin{macro}{\nmv@citex@}
% The new version of \cs{@citex} is needed that looks ahead of the
% citation using \cs{futurelet}.  There are three arguments to
% \cs{@citex} when using \pkg{natbib}.  Other than that, the trick
% used here is similar to that in \pkg{cite}.
%    \begin{macrocode}
\def\nmv@citex@[#1][#2]#3{%
  \leavevmode
  \skip@\lastskip
  \unskip
  \begingroup
%    \end{macrocode}
%\begin{macro}{\nmv@arg}
% The arguments of the macro now need to be saved, before handing off
% to the ``search'' macro.
%    \begin{macrocode}
    \def\nmv@arg{[#1][#2]{#3}}%
%    \end{macrocode}
%\end{macro}
%\begin{macro}{\nmv@citex@end}
% The entire block to be executed after punctuation is set up here,
% to make recursion easier.  The system used is very similar to that
% in \pkg{cite}.  Notice that the \cs{relax} is essential.
%    \begin{macrocode}
    \newcommand*{\nmv@citex@end}{%
        \expandafter\nmv@natcitex\nmv@arg
      \endgroup}%
    \global\nmv@citefalse
    \nmv@citex@getnext\relax}
%    \end{macrocode}
%\end{macro}
%\begin{macro}{\nmv@citex@getnext}
%\begin{macro}{\nmv@citex@next}
% The next token on the input stack is saved into
% \cs{nmv@citex@next}, after gobbling up one token.
%    \begin{macrocode}
\newcommand*{\nmv@citex@getnext}[1]{%
  \futurelet\nmv@citex@next\nmv@citex@punct}
%    \end{macrocode}
%\end{macro}
%\end{macro}
%\end{macro}
%\begin{macro}{\nmv@citex@punct}
%\begin{macro}{\nmv@citex@loop}
% The working macro for moving the punctuation.  This is very much
% like \cs{@citey} in the \pkg{cite} package. The initial assumption
% is that the loop will terminate, and so the recursion call will
% simply do the finalisation.
%    \begin{macrocode}
\newcommand*{\nmv@citex@punct}{%
  \let\nmv@citex@loop\nmv@citex@end
%    \end{macrocode}
% A check is made for doubled full stops.
%    \begin{macrocode}
  \ifx.\nmv@citex@next\@empty
    \ifnum\spacefactor<\nmv@citex@sfac\else
      \let\nmv@citex@next\relax
      \let\nmv@citex@loop\nmv@citex@getnext
    \fi
  \fi
%    \end{macrocode}
% The other cases are handled.
%    \begin{macrocode}
  \expandafter\nmv@citex@pnct\natmovechars\@empty
  \nmv@citex@loop}
%    \end{macrocode}
%\end{macro}
%\end{macro}
%\begin{macro}{\nmv@citex@pnct}
% The final part of the punctuation moving system.
%    \begin{macrocode}
\def\nmv@citex@pnct#1#2\@empty{%
  \ifx\nmv@citex@next#1\@empty
    #1%
    \let\nmv@citex@loop\nmv@citex@getnext
  \fi
  \ifx\@empty#2\@empty\else
    \expandafter\nmv@citex@pnct#2\@empty
  \fi}
%    \end{macrocode}
%\end{macro}
%\begin{macro}{\nmv@citex@sfac}
% The value of the spacing factor after a full stop is used to signal
% doubled punctuation.  For French spacing, a bit of patching is
% needed.
%    \begin{macrocode}
\mathchardef\nmv@citex@sfac3000
\expandafter\renewcommand\expandafter*\expandafter{\expandafter%
  \frenchspacing\expandafter}\expandafter{%
  \frenchspacing
  \mathchardef\nmv@citex@sfac1001
  \sfcode`\.\nmv@citex@sfac
  \sfcode`\?\nmv@citex@sfac
  \sfcode`\!\nmv@citex@sfac}
\ifnum\sfcode`\.=\@m
  \frenchspacing
\fi
%    \end{macrocode}
%\end{macro}
%\begin{macro}{\nmv@activate}
%\begin{macro}{\@citex}
%\begin{macro}{\nmv@natcitex}
%\begin{macro}{\cite}
%\begin{macro}{\nmv@cite}
%\begin{macro}{\nmv@notactivate}
% The swap has to be done at the beginning of the document. The
% internal flag from \pkg{natbib} is used, but under the
% circumstances we should be safe.  \cs{cite} is also patched to make
% the system active.  A
%    \begin{macrocode}
\newcommand*{\nmv@activate}{%
  \let\nmv@natcitex\@citex
  \let\@citex\nmv@citex
  \let\nmv@cite\cite
  \renewcommand*{\cite}[2][]{%
    \nmv@ifmtarg{##1}
      {\nmv@citetrue
       \nmv@cite{##2}}
      {\nmv@citefalse
       \nmv@cite[##1]{##2}}}}
\AtBeginDocument{
  \@ifpackageloaded{natbib}
    {\ifNAT@super
      \nmv@activate
     \else
      \nmv@notactivate
     \fi}
    {\PackageWarning{natmove}
       {The natbib package has not been loaded}}}
\newcommand*{\nmv@notactivate}{}
%    \end{macrocode}
%\end{macro}
%\end{macro}
%\end{macro}
%\end{macro}
%\end{macro}
%\end{macro}
%\begin{macro}{\natmovechars}
% A user macro is needed for moving characters.
%    \begin{macrocode}
\newcommand*{\natmovechars}{.,;:}
%    \end{macrocode}
%\end{macro}
%\iffalse
%</package>
%\fi
%
%\Finale
%
%\iffalse
%<*jawltxdoc>
\NeedsTeXFormat{LaTeX2e}
\ProvidesPackage{jawltxdoc}
\usepackage[T1]{fontenc}
\usepackage{lmodern}
\usepackage[final]{listings,graphicx,microtype}
\usepackage[scaled=0.95]{helvet}
\usepackage[version=3]{mhchem}
\usepackage[osf]{mathpazo}
\usepackage{booktabs,array,url,courier,xspace,varioref}
\usepackage{upgreek,ifpdf,float,caption,longtable,babel}
\begingroup
  \@ifundefined{eTeXversion}
    {\aftergroup\@gobble}
    {\aftergroup\@firstofone}
\endgroup
  {\usepackage{etoolbox}}
\floatstyle{plaintop}
\restylefloat{table}
\labelformat{figure}{\figurename~#1}
\labelformat{table}{\tablename~#1}
\ifpdf
  \usepackage{embedfile}
  \embedfile[%
    stringmethod=escape,%
    mimetype=plain/text,%
    desc={LaTeX docstrip source archive for package `\jobname'}%
    ]{\jobname.dtx}
\fi
\IfFileExists{\jobname.sty}
  {\usepackage{\jobname}}{}
\usepackage[numbered]{hypdoc}
\setcounter{IndexColumns}{2}
\newlength\LaTeXwidth
\newlength\LaTeXoutdent
\newlength\LaTeXgap
\setlength\LaTeXgap{1em}
\setlength\LaTeXoutdent{-0.15\textwidth}
\newbox\lst@samplebox
\edef\LaTeXexamplefile{\jobname.tmp}
\lst@RequireAspects{writefile}
\lstnewenvironment{LaTeXexample}[1][example]{%
  \global\let\lst@intname\@empty
  \ifcsname LaTeXcode#1\endcsname
    \expandafter\let\expandafter\LaTeXcode
      \csname LaTeXcode#1\endcsname
    \expandafter\let\expandafter\LaTeXcodeend
      \csname LaTeXcode#1end\endcsname
  \else
    \PackageError{jawltxdoc}
      {Undefined example type `#1'}
      \@ehd
    \let\LaTeXcode\relax
    \let\LaTeXcodeend\relax
  \fi
  \LaTeXcode}
  {\lst@EndWriteFile
   \LaTeXcodeend}
\newcommand*{\LaTeXcodeexample}{%
  \setbox\lst@samplebox=\hbox\bgroup
  \LaTeXcodefloat}
\let\LaTeXcoderesultonly\LaTeXcodeexample
\newcommand*{\LaTeXcodeexampleend}{%
  \egroup
  \setlength\LaTeXwidth{\wd\lst@samplebox}%
  \begin{list}{}{%
    \setlength\itemindent{0pt}
    \setlength\leftmargin\LaTeXoutdent
    \setlength\rightmargin{0pt}}%
    \item
      \setlength\LaTeXoutdent{-0.15\textwidth}
      \begin{minipage}[c]{%
        \textwidth-\LaTeXwidth-\LaTeXoutdent-\LaTeXgap}
        \LaTeXcodefloatend
      \end{minipage}%
      \hfill
      \begin{minipage}[c]{\LaTeXwidth}%
        \hbox to\linewidth{\box\lst@samplebox\hss}%
      \end{minipage}%
  \end{list}}
\newcommand*{\LaTeXcodefloat}{%
  \setkeys{lst}{tabsize=4,gobble=3,breakindent=0pt,
    basicstyle=\small\ttfamily,basewidth=0.51em,
    keywordstyle=\color{blue}}%
  \lst@BeginAlsoWriteFile{\LaTeXexamplefile}}
\let\LaTeXcodenoexample\LaTeXcodefloat
\let\LaTeXcodenoexampleend\@empty
\newcommand*{\LaTeXcodefloatend}{%
  \MakePercentComment\catcode`\^^M=10\relax
  \small
  {\setkeys{lst}{SelectCharTable=\lst@ReplaceInput{\^\^I}%
    {\lst@ProcessTabulator}}%
    \leavevmode \input{\LaTeXexamplefile}}%
  \MakePercentIgnore}
\newcommand*{\LaTeXcoderesultonlyend}{\egroup\LaTeXcodefloatend}
\lstnewenvironment{BibTeXexample}{%
  \global\let\lst@intname\@empty
  \setbox\lst@samplebox=\hbox\bgroup
  \setkeys{lst}{tabsize=4,gobble=3,breakindent=0pt,
    basicstyle=\small\ttfamily,basewidth=0.51em,
    keywordstyle=\color{black}}
  \lst@BeginAlsoWriteFile{\LaTeXexamplefile}}
 {\lst@EndWriteFile
   \LaTeXcodeexampleend}
\newcommand*{\DescribeOption}{%
  \leavevmode\@bsphack\begingroup\MakePrivateLetters
  \Describe@Option}
\newcommand*{\Describe@Option}[1]{\endgroup
              \marginpar{\raggedleft\PrintDescribeEnv{#1}}%
              \SpecialOptionIndex{#1}\@esphack\ignorespaces}
\newcommand*{\SpecialOptionIndex}[1]{\@bsphack
    \index{#1\actualchar{\protect\ttfamily#1}
           (option)\encapchar usage}%
    \index{options:\levelchar#1\actualchar{\protect\ttfamily#1}%
      \encapchar usage}\@esphack}
\newcommand*{\indexopt}[1]{\DescribeOption{#1}\opt{#1}}
\newcommand*{\DescribeOptionInfo}[2]{%
  \DescribeOption{#1}%
  \opt{#1=\meta{#2}}\xspace}
\newcommand*{\ofixarg}[1]{%
  {\ttfamily[}%
  \ifmmode \expandafter \nfss@text \fi
  {%
    \meta@font@select
    \edef\meta@hyphen@restore{%
      \hyphenchar\the\font\the\hyphenchar\font}%
    \hyphenchar\font\m@ne
    \language\l@nohyphenation
    #1\/%
    \meta@hyphen@restore
    }%
    {\ttfamily]}}
\newcommand*{\pkg}[1]{\textsf{#1}}
\newcommand*{\currpkg}{\pkg{\jobname}\xspace}
\newcommand*{\opt}[1]{\texttt{#1}}
\newcommand*{\defaultopt}[1]{\opt{\textbf{#1}}}
\newcommand*{\file}[1]{\texttt{#1}}
\newcommand*{\ext}[1]{\file{.#1}}
\newcommand*{\latin}[1]{\emph{#1}}
\newcommand*{\etc}{%
  \@ifnextchar.
    {\latin{etc}}
    {\latin{etc}.\xspace}}
\newcommand*{\eg}{%
  \@ifnextchar.
    {\latin{e.g}}
    {\latin{e.g}.\xspace}}
\newcommand*{\ie}{%
  \@ifnextchar.
    {\latin{i.e}}
    {\latin{i.e}.\xspace}}
\newcommand*{\etal}{%
  \@ifnextchar.
    {\latin{et~al.}}
    {\latin{et~al}.\xspace}}
\newcommand*{\AMS}{{\protect\usefont{OMS}{cmsy}{m}{n}%
  A\kern-.1667em\lower.5ex\hbox{M}\kern-.125emS}}
\providecommand*{\eTeX}{\ensuremath{\varepsilon}-\TeX}
\DeclareRobustCommand*{\XeTeX}
  {X\kern-.125em\lower.5ex\hbox{\reflectbox{E}}\kern-.1667em\TeX}
\providecommand*{\CTAN}{\textsc{ctan}}
\@ifpackageloaded{etoolbox}
  {\patchcmd{\@addmarginpar}
    {\@latex@warning@no@line {Marginpar on page \thepage\space moved}}
    {\relax}{}{}}
  {}
\newcounter{argument}
\g@addto@macro\endmacro{\setcounter{argument}{0}}
\newcommand*\darg[1]{%
  \stepcounter{argument}%
  {\ttfamily\char`\#\theargument~:~}#1\par\noindent\ignorespaces}
\newcommand*\doarg[1]{%
  \stepcounter{argument}%
  {\ttfamily\makebox[0pt][r]{[}%
   \char`\#\theargument]:~}#1\par\noindent\ignorespaces}
%</jawltxdoc>
%\fi

%
% Documentation:
%    (a) Without write18 enabled:
%          pdflatex natmove.dtx
%          bibtex8 --wolfgang natmove
%          makeindex -s gind.ist natmove.idx
%          makeindex -s gglo.ist -o natmove.gls natmove.glo
%          pdflatex natmove.dtx
%          pdflatex natmove.dtx
%    (b) With write18 enabled:
%          pdflatex natmove.dtx
%          pdflatex natmove.dtx
%          pdflatex natmove.dtx
%
% Installation:
%     Copy natmove.sty to a location searched by TeX, and if required
%     by your TeX installation, run the appropriate command to build
%     a hash of files (texhash, initexmf --update-fndb, etc.)
%
% Note:
%     The jawltxdoc.sty file is not needed for installation,
%     only for building the documentation; it may be deleted
%     after producing the documentation (if necessary).
%
%<*ignore>
% This is all taken verbatim from Heiko Oberdiek's packages
\begingroup
  \def\x{LaTeX2e}%
\expandafter\endgroup
\ifcase 0\ifx\install y1\fi\expandafter
         \ifx\csname processbatchFile\endcsname\relax\else1\fi
         \ifx\fmtname\x\else 1\fi\relax
\else\csname fi\endcsname
%</ignore>
%<*install>
\input docstrip.tex
\keepsilent
\askforoverwritefalse
\preamble
 ----------------------------------------------------------------
 natmove --- Automatic citation moving with natbib
 Maintained by Joseph Wright
 E-mail: joseph.wright@morningstar2.co.uk
 Released under the LaTeX Project Public License v1.3c or later
 See http://www.latex-project.org/lppl.txt
 ----------------------------------------------------------------

\endpreamble
\Msg{Generating natmove files:}
\generate{\file{jawltxdoc.sty}{\from{\jobname.dtx}{jawltxdoc}}
}
\usedir{tex/latex/achemso}
\generate{\file{\jobname.sty}{\from{\jobname.dtx}{package}}
}
\usedir{source/latex/achemso}
\generate{\file{\jobname.ins}{\from{\jobname.dtx}{install}}
}
\endbatchfile
%</install>
%<*ignore>
\fi
% Will Robertson's trick
\immediate\write18{bibtex8 --wolfgang \jobname}
\immediate\write18{makeindex -s gind.ist -o \jobname.ind  \jobname.idx}
\immediate\write18{makeindex -s gglo.ist -o \jobname.gls  \jobname.glo}
%</ignore>
%<*driver>
\PassOptionsToClass{a4paper}{article}
\documentclass[german,english,UKenglish]{ltxdoc}
\EnableCrossrefs
\CodelineIndex
\RecordChanges
%\OnlyDescription
\usepackage[sort&compress,numbers,super]{natbib}
\usepackage{mciteplus}
\usepackage{jawltxdoc}
\begin{document}
  \DocInput{\jobname.dtx}
\end{document}
%</driver>
% \fi
%
%\CheckSum{158}
%
% \CharacterTable
%  {Upper-case    \A\B\C\D\E\F\G\H\I\J\K\L\M\N\O\P\Q\R\S\T\U\V\W\X\Y\Z
%   Lower-case    \a\b\c\d\e\f\g\h\i\j\k\l\m\n\o\p\q\r\s\t\u\v\w\x\y\z
%   Digits        \0\1\2\3\4\5\6\7\8\9
%   Exclamation   \!     Double quote  \"     Hash (number) \#
%   Dollar        \$     Percent       \%     Ampersand     \&
%   Acute accent  \'     Left paren    \(     Right paren   \)
%   Asterisk      \*     Plus          \+     Comma         \,
%   Minus         \-     Point         \.     Solidus       \/
%   Colon         \:     Semicolon     \;     Less than     \<
%   Equals        \=     Greater than  \>     Question mark \?
%   Commercial at \@     Left bracket  \[     Backslash     \\
%   Right bracket \]     Circumflex    \^     Underscore    \_
%   Grave accent  \`     Left brace    \{     Vertical bar  \|
%   Right brace   \}     Tilde         \~}
%
%\GetFileInfo{\jobname.sty}
%
%\DoNotIndex{\!,\.,\?,\@empty,\@firstoftwo,\@ifmtarg}
%\DoNotIndex{\@ifpackageloaded,\@m,\@nil,\@secondoftwo}
%\DoNotIndex{\AtBeginDocument,\begingroup,\catcode,\def,\else}
%\DoNotIndex{\endgroup,\expandafter,\fi,\frenchspacing,\futurelet}
%\DoNotIndex{\gdef,\global,\ifnum,\ifx,\ifNAT@super,\lastskip}
%\DoNotIndex{\leavevmode,\let,\long,\mathchardef,\NeedsTeXFormat}
%\DoNotIndex{,\newcommand,\newif,\PackageWarning,\ProvidesPackage,\Q}
%\DoNotIndex{\relax,\renewcommand,\RequirePackage,\sfcode,\skip@}
%\DoNotIndex{\space,\spacefactor,\unskip}
%
%\changes{v1.0}{2008/06/22}{First public release}
%
%\setkeys{lst}{language=[LaTeX]{TeX},moretexcs={citenum,
%  citeyear,citeauthor}}
%
%\title{\currpkg\ ---  Automatic citation moving with natbib^^A
%  \thanks{This file describes version \fileversion, last revised
%    \filedate.}}
%\author{Joseph Wright^^A
%  \thanks{E-mail: joseph.wright@morningstar2.co.uk}}
%\date{Released \filedate}
%
%\maketitle
%
%\newcommand*{\ACS}{\textsc{acs}}
%\begin{abstract}
% The \currpkg package adds the ability to move citations after
% punctuation automatically to the \pkg{natbib} package, in the same
% way as with \pkg{cite}.  With non-superscript citations, no action
% is taken.
%\end{abstract}
%
%\begin{multicols}{2}
%  \tableofcontents
%\end{multicols}
%
%\section{Using the package}
% The \currpkg package does only one job.  It brings the ability to
% move punctuation after citations, using code borrowed from the
% \pkg{cite} package.
%\begin{LaTeXexample}
%  Some text \cite{Coghill2006} some more text.\\
%  Some text ending a sentence \cite{Coghill2006}.
%\end{LaTeXexample}
% This is deactivated for other citation types.
%\begin{LaTeXexample}
%  Some text \citeyear{Coghill2006}.\\
%  Some text \citeauthor{Coghill2006}.\\
%  Some text \citenum{Coghill2006}.
%\end{LaTeXexample}
% The package does nothing if the \opt{super} option has not been
% given to \pkg{natbib}.  This means that the source can be written
% without needing to decide where citations will to appear, with the
% \opt{super} option for \pkg{natbib} controlling the result.
%
%\DescribeMacro{\natmovechars}
% One user macro is provided: \cs{natmovechars}.  This contains
% the characters which are moved before superscript punctuation.
% The default contents is |,;:.| and can be set using
% \cs{renewcommand*}:
%\begin{LaTeXexample}
%  \renewcommand*{\natmovechars}{.}
%  Some text \cite{Coghill2006},
%  more text \cite{Coghill2006}.
%\end{LaTeXexample}
%
%\StopEventually{%
%  \PrintChanges
%  \PrintIndex
%  \bibliographystyle{achemso}
%  \bibliography{achemso}}
%
%\iffalse
%<*package>
%\fi
%\section{The code}
%\begin{macro}{\nmv@id}
% The package file is designed to be usable with any document class.
% It sets up the basics, but leaves some settings to the class file.
%    \begin{macrocode}
\NeedsTeXFormat{LaTeX2e}
\def\nmv@id$#1: #2.#3 #4 #5-#6-#7 #8 #9${%
  #5/#6/#7\space v1.0\space}
\ProvidesPackage{natmove}
  [\nmv@id$Id: natmove.dtx 29 2008-08-22 07:47:26Z joseph $
   Automatic citation moving with natbib]
%    \end{macrocode}
%\end{macro}
%\begin{macro}{\nmv@ifmtarg}
%\begin{macro}{\nmv@xifmtarg}
% To keep down dependance on other packages, the very short code
% block from \pkg{ifmtarg} is copied here with an internal name.
%    \begin{macrocode}
\begingroup
  \catcode`\Q=3
  \long\gdef\nmv@ifmtarg#1{%
    \nmv@xifmtarg#1QQ\@secondoftwo\@firstoftwo\@nil}
  \long\gdef\nmv@xifmtarg#1#2Q#3#4#5\@nil{#4}
\endgroup
%    \end{macrocode}
%\end{macro}
%\end{macro}
%\begin{macro}{\ifnmv@cite}
% A flag is need to watch whether \cs{cite} or another macro is in
% use.
%    \begin{macrocode}
\newif\ifnmv@cite
%    \end{macrocode}
%\end{macro}
%\begin{macro}{\nmv@citex}
% Using the flag, either the new internal macro, or the
% \pkg{natbib} original, can be called.
%    \begin{macrocode}
\newcommand*{\nmv@citex}{%
  \ifnmv@cite
    \expandafter\nmv@citex@
  \else
    \expandafter\nmv@natcitex
  \fi}
%    \end{macrocode}
%\end{macro}
%\begin{macro}{\nmv@citex@}
% The new version of \cs{@citex} is needed that looks ahead of the
% citation using \cs{futurelet}.  There are three arguments to
% \cs{@citex} when using \pkg{natbib}.  Other than that, the trick
% used here is similar to that in \pkg{cite}.
%    \begin{macrocode}
\def\nmv@citex@[#1][#2]#3{%
  \leavevmode
  \skip@\lastskip
  \unskip
  \begingroup
%    \end{macrocode}
%\begin{macro}{\nmv@arg}
% The arguments of the macro now need to be saved, before handing off
% to the ``search'' macro.
%    \begin{macrocode}
    \def\nmv@arg{[#1][#2]{#3}}%
%    \end{macrocode}
%\end{macro}
%\begin{macro}{\nmv@citex@end}
% The entire block to be executed after punctuation is set up here,
% to make recursion easier.  The system used is very similar to that
% in \pkg{cite}.  Notice that the \cs{relax} is essential.
%    \begin{macrocode}
    \newcommand*{\nmv@citex@end}{%
        \expandafter\nmv@natcitex\nmv@arg
      \endgroup}%
    \global\nmv@citefalse
    \nmv@citex@getnext\relax}
%    \end{macrocode}
%\end{macro}
%\begin{macro}{\nmv@citex@getnext}
%\begin{macro}{\nmv@citex@next}
% The next token on the input stack is saved into
% \cs{nmv@citex@next}, after gobbling up one token.
%    \begin{macrocode}
\newcommand*{\nmv@citex@getnext}[1]{%
  \futurelet\nmv@citex@next\nmv@citex@punct}
%    \end{macrocode}
%\end{macro}
%\end{macro}
%\end{macro}
%\begin{macro}{\nmv@citex@punct}
%\begin{macro}{\nmv@citex@loop}
% The working macro for moving the punctuation.  This is very much
% like \cs{@citey} in the \pkg{cite} package. The initial assumption
% is that the loop will terminate, and so the recursion call will
% simply do the finalisation.
%    \begin{macrocode}
\newcommand*{\nmv@citex@punct}{%
  \let\nmv@citex@loop\nmv@citex@end
%    \end{macrocode}
% A check is made for doubled full stops.
%    \begin{macrocode}
  \ifx.\nmv@citex@next\@empty
    \ifnum\spacefactor<\nmv@citex@sfac\else
      \let\nmv@citex@next\relax
      \let\nmv@citex@loop\nmv@citex@getnext
    \fi
  \fi
%    \end{macrocode}
% The other cases are handled.
%    \begin{macrocode}
  \expandafter\nmv@citex@pnct\natmovechars\@empty
  \nmv@citex@loop}
%    \end{macrocode}
%\end{macro}
%\end{macro}
%\begin{macro}{\nmv@citex@pnct}
% The final part of the punctuation moving system.
%    \begin{macrocode}
\def\nmv@citex@pnct#1#2\@empty{%
  \ifx\nmv@citex@next#1\@empty
    #1%
    \let\nmv@citex@loop\nmv@citex@getnext
  \fi
  \ifx\@empty#2\@empty\else
    \expandafter\nmv@citex@pnct#2\@empty
  \fi}
%    \end{macrocode}
%\end{macro}
%\begin{macro}{\nmv@citex@sfac}
% The value of the spacing factor after a full stop is used to signal
% doubled punctuation.  For French spacing, a bit of patching is
% needed.
%    \begin{macrocode}
\mathchardef\nmv@citex@sfac3000
\expandafter\renewcommand\expandafter*\expandafter{\expandafter%
  \frenchspacing\expandafter}\expandafter{%
  \frenchspacing
  \mathchardef\nmv@citex@sfac1001
  \sfcode`\.\nmv@citex@sfac
  \sfcode`\?\nmv@citex@sfac
  \sfcode`\!\nmv@citex@sfac}
\ifnum\sfcode`\.=\@m
  \frenchspacing
\fi
%    \end{macrocode}
%\end{macro}
%\begin{macro}{\nmv@activate}
%\begin{macro}{\@citex}
%\begin{macro}{\nmv@natcitex}
%\begin{macro}{\cite}
%\begin{macro}{\nmv@cite}
%\begin{macro}{\nmv@notactivate}
% The swap has to be done at the beginning of the document. The
% internal flag from \pkg{natbib} is used, but under the
% circumstances we should be safe.  \cs{cite} is also patched to make
% the system active.  A
%    \begin{macrocode}
\newcommand*{\nmv@activate}{%
  \let\nmv@natcitex\@citex
  \let\@citex\nmv@citex
  \let\nmv@cite\cite
  \renewcommand*{\cite}[2][]{%
    \nmv@ifmtarg{##1}
      {\nmv@citetrue
       \nmv@cite{##2}}
      {\nmv@citefalse
       \nmv@cite[##1]{##2}}}}
\AtBeginDocument{
  \@ifpackageloaded{natbib}
    {\ifNAT@super
      \nmv@activate
     \else
      \nmv@notactivate
     \fi}
    {\PackageWarning{natmove}
       {The natbib package has not been loaded}}}
\newcommand*{\nmv@notactivate}{}
%    \end{macrocode}
%\end{macro}
%\end{macro}
%\end{macro}
%\end{macro}
%\end{macro}
%\end{macro}
%\begin{macro}{\natmovechars}
% A user macro is needed for moving characters.
%    \begin{macrocode}
\newcommand*{\natmovechars}{.,;:}
%    \end{macrocode}
%\end{macro}
%\iffalse
%</package>
%\fi
%
%\Finale
%
%\iffalse
%<*jawltxdoc>
\NeedsTeXFormat{LaTeX2e}
\ProvidesPackage{jawltxdoc}
\usepackage[T1]{fontenc}
\usepackage{lmodern}
\usepackage[final]{listings,graphicx,microtype}
\usepackage[scaled=0.95]{helvet}
\usepackage[version=3]{mhchem}
\usepackage[osf]{mathpazo}
\usepackage{booktabs,array,url,courier,xspace,varioref}
\usepackage{upgreek,ifpdf,float,caption,longtable,babel}
\begingroup
  \@ifundefined{eTeXversion}
    {\aftergroup\@gobble}
    {\aftergroup\@firstofone}
\endgroup
  {\usepackage{etoolbox}}
\floatstyle{plaintop}
\restylefloat{table}
\labelformat{figure}{\figurename~#1}
\labelformat{table}{\tablename~#1}
\ifpdf
  \usepackage{embedfile}
  \embedfile[%
    stringmethod=escape,%
    mimetype=plain/text,%
    desc={LaTeX docstrip source archive for package `\jobname'}%
    ]{\jobname.dtx}
\fi
\IfFileExists{\jobname.sty}
  {\usepackage{\jobname}}{}
\usepackage[numbered]{hypdoc}
\setcounter{IndexColumns}{2}
\newlength\LaTeXwidth
\newlength\LaTeXoutdent
\newlength\LaTeXgap
\setlength\LaTeXgap{1em}
\setlength\LaTeXoutdent{-0.15\textwidth}
\newbox\lst@samplebox
\edef\LaTeXexamplefile{\jobname.tmp}
\lst@RequireAspects{writefile}
\lstnewenvironment{LaTeXexample}[1][example]{%
  \global\let\lst@intname\@empty
  \ifcsname LaTeXcode#1\endcsname
    \expandafter\let\expandafter\LaTeXcode
      \csname LaTeXcode#1\endcsname
    \expandafter\let\expandafter\LaTeXcodeend
      \csname LaTeXcode#1end\endcsname
  \else
    \PackageError{jawltxdoc}
      {Undefined example type `#1'}
      \@ehd
    \let\LaTeXcode\relax
    \let\LaTeXcodeend\relax
  \fi
  \LaTeXcode}
  {\lst@EndWriteFile
   \LaTeXcodeend}
\newcommand*{\LaTeXcodeexample}{%
  \setbox\lst@samplebox=\hbox\bgroup
  \LaTeXcodefloat}
\let\LaTeXcoderesultonly\LaTeXcodeexample
\newcommand*{\LaTeXcodeexampleend}{%
  \egroup
  \setlength\LaTeXwidth{\wd\lst@samplebox}%
  \begin{list}{}{%
    \setlength\itemindent{0pt}
    \setlength\leftmargin\LaTeXoutdent
    \setlength\rightmargin{0pt}}%
    \item
      \setlength\LaTeXoutdent{-0.15\textwidth}
      \begin{minipage}[c]{%
        \textwidth-\LaTeXwidth-\LaTeXoutdent-\LaTeXgap}
        \LaTeXcodefloatend
      \end{minipage}%
      \hfill
      \begin{minipage}[c]{\LaTeXwidth}%
        \hbox to\linewidth{\box\lst@samplebox\hss}%
      \end{minipage}%
  \end{list}}
\newcommand*{\LaTeXcodefloat}{%
  \setkeys{lst}{tabsize=4,gobble=3,breakindent=0pt,
    basicstyle=\small\ttfamily,basewidth=0.51em,
    keywordstyle=\color{blue}}%
  \lst@BeginAlsoWriteFile{\LaTeXexamplefile}}
\let\LaTeXcodenoexample\LaTeXcodefloat
\let\LaTeXcodenoexampleend\@empty
\newcommand*{\LaTeXcodefloatend}{%
  \MakePercentComment\catcode`\^^M=10\relax
  \small
  {\setkeys{lst}{SelectCharTable=\lst@ReplaceInput{\^\^I}%
    {\lst@ProcessTabulator}}%
    \leavevmode \input{\LaTeXexamplefile}}%
  \MakePercentIgnore}
\newcommand*{\LaTeXcoderesultonlyend}{\egroup\LaTeXcodefloatend}
\lstnewenvironment{BibTeXexample}{%
  \global\let\lst@intname\@empty
  \setbox\lst@samplebox=\hbox\bgroup
  \setkeys{lst}{tabsize=4,gobble=3,breakindent=0pt,
    basicstyle=\small\ttfamily,basewidth=0.51em,
    keywordstyle=\color{black}}
  \lst@BeginAlsoWriteFile{\LaTeXexamplefile}}
 {\lst@EndWriteFile
   \LaTeXcodeexampleend}
\newcommand*{\DescribeOption}{%
  \leavevmode\@bsphack\begingroup\MakePrivateLetters
  \Describe@Option}
\newcommand*{\Describe@Option}[1]{\endgroup
              \marginpar{\raggedleft\PrintDescribeEnv{#1}}%
              \SpecialOptionIndex{#1}\@esphack\ignorespaces}
\newcommand*{\SpecialOptionIndex}[1]{\@bsphack
    \index{#1\actualchar{\protect\ttfamily#1}
           (option)\encapchar usage}%
    \index{options:\levelchar#1\actualchar{\protect\ttfamily#1}%
      \encapchar usage}\@esphack}
\newcommand*{\indexopt}[1]{\DescribeOption{#1}\opt{#1}}
\newcommand*{\DescribeOptionInfo}[2]{%
  \DescribeOption{#1}%
  \opt{#1=\meta{#2}}\xspace}
\newcommand*{\ofixarg}[1]{%
  {\ttfamily[}%
  \ifmmode \expandafter \nfss@text \fi
  {%
    \meta@font@select
    \edef\meta@hyphen@restore{%
      \hyphenchar\the\font\the\hyphenchar\font}%
    \hyphenchar\font\m@ne
    \language\l@nohyphenation
    #1\/%
    \meta@hyphen@restore
    }%
    {\ttfamily]}}
\newcommand*{\pkg}[1]{\textsf{#1}}
\newcommand*{\currpkg}{\pkg{\jobname}\xspace}
\newcommand*{\opt}[1]{\texttt{#1}}
\newcommand*{\defaultopt}[1]{\opt{\textbf{#1}}}
\newcommand*{\file}[1]{\texttt{#1}}
\newcommand*{\ext}[1]{\file{.#1}}
\newcommand*{\latin}[1]{\emph{#1}}
\newcommand*{\etc}{%
  \@ifnextchar.
    {\latin{etc}}
    {\latin{etc}.\xspace}}
\newcommand*{\eg}{%
  \@ifnextchar.
    {\latin{e.g}}
    {\latin{e.g}.\xspace}}
\newcommand*{\ie}{%
  \@ifnextchar.
    {\latin{i.e}}
    {\latin{i.e}.\xspace}}
\newcommand*{\etal}{%
  \@ifnextchar.
    {\latin{et~al.}}
    {\latin{et~al}.\xspace}}
\newcommand*{\AMS}{{\protect\usefont{OMS}{cmsy}{m}{n}%
  A\kern-.1667em\lower.5ex\hbox{M}\kern-.125emS}}
\providecommand*{\eTeX}{\ensuremath{\varepsilon}-\TeX}
\DeclareRobustCommand*{\XeTeX}
  {X\kern-.125em\lower.5ex\hbox{\reflectbox{E}}\kern-.1667em\TeX}
\providecommand*{\CTAN}{\textsc{ctan}}
\@ifpackageloaded{etoolbox}
  {\patchcmd{\@addmarginpar}
    {\@latex@warning@no@line {Marginpar on page \thepage\space moved}}
    {\relax}{}{}}
  {}
\newcounter{argument}
\g@addto@macro\endmacro{\setcounter{argument}{0}}
\newcommand*\darg[1]{%
  \stepcounter{argument}%
  {\ttfamily\char`\#\theargument~:~}#1\par\noindent\ignorespaces}
\newcommand*\doarg[1]{%
  \stepcounter{argument}%
  {\ttfamily\makebox[0pt][r]{[}%
   \char`\#\theargument]:~}#1\par\noindent\ignorespaces}
%</jawltxdoc>
%\fi
